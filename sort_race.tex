\documentclass[AMA,STIX1COL]{WileyNJD-v2}
\usepackage{moreverb}


\newcommand\BibTeX{{\rmfamily B\kern-.05em \textsc{i\kern-.025em b}\kern-.08em
T\kern-.1667em\lower.7ex\hbox{E}\kern-.125emX}}

\newcommand{\klit}{\textcolor{blue}{\emph{k-limited}}}
\newcommand{\keq}{\textcolor{blue}{\emph{k-equal teeth}}}
\newcommand{\kev}{\textcolor{blue}{\emph{k-even teeth}}}
\newcommand{\ksha}{\textcolor{blue}{\emph{k-sharp teeth}}}
\newcommand{\kshu}{\textcolor{blue}{\emph{k-shuffled teeth}}}
\newcommand{\kdis}{\textcolor{blue}{\emph{k-distance}}}
\newcommand{\kexc}{\textcolor{blue}{\emph{k-exchange}}}

\newcommand {\mono}{\emph{\textcolor{blue}{Mono(A)}}} 

\newcommand{\qusort}{\emph{quicksort }}
\newcommand{\qusortn}{\emph{quicksort}}
\newcommand{\qsort}{\emph{qsort }}
\newcommand{\qsortn}{\emph{qsort}}
\newcommand{\tsort}{\emph{timsort }}
\newcommand{\tsortn}{\emph{timsort}}
\newcommand{\msort}{\emph{mergesort }}
\newcommand{\msortn}{\emph{mergesort}}

\usepackage{xcolor}
\usepackage{listings}

\definecolor{mGreen}{rgb}{0,0.6,0}
\definecolor{mGray}{rgb}{0.5,0.5,0.5}
\definecolor{mPurple}{rgb}{0.58,0,0.82}
\definecolor{backgroundColour}{rgb}{0.95,0.95,0.92}

\lstdefinestyle{CStyle}{
    backgroundcolor=\color{backgroundColour},   
    commentstyle=\color{mGreen},
    keywordstyle=\color{magenta},
    numberstyle=\tiny\color{mGray},
    stringstyle=\color{mPurple},
    basicstyle=\footnotesize,
    breakatwhitespace=false,         
    breaklines=true,                 
    captionpos=b,                    
    keepspaces=true,                 
    numbers=left,                    
    numbersep=5pt,                  
    showspaces=false,                
    showstringspaces=false,
    showtabs=false,                  
    tabsize=2,
    language=C
}

\usepackage{color}

\definecolor{dkgreen}{rgb}{0,0.6,0}
\definecolor{gray}{rgb}{0.5,0.5,0.5}
\definecolor{mauve}{rgb}{0.58,0,0.82}

\lstset{frame=tb,
  language=Java,
  aboveskip=3mm,
  belowskip=3mm,
  showstringspaces=false,
  columns=flexible,
  basicstyle={\small\ttfamily},
  numbers=none,
  numberstyle=\tiny\color{gray},
  keywordstyle=\color{blue},
  commentstyle=\color{dkgreen},
  stringstyle=\color{mauve},
  breaklines=true,
  breakatwhitespace=true,
  tabsize=3
}

\articletype{Article Type}%

\received{<day> <Month>, <year>}
\revised{<day> <Month>, <year>}
\accepted{<day> <Month>, <year>}

%\raggedbottom

\begin{document}

\title{Sort Race}

\author[1]{Hantao Zhang*}

\author[2]{Baoluo Meng}

\author[3]{Yiwen Liang}

\authormark{Zhang \textsc{et. al.}}


\address[1]{\orgdiv{Department of Computer Science}, \orgname{The University of Iowa}, \orgaddress{\state{Iowa}, \country{USA}}}

\address[2]{\orgname{GE Research}, \orgaddress{\state{New York}, \country{USA}}}

\address[3]{\orgdiv{Department of Computer Science}, \orgname{Wuhan University}, \orgaddress{\state{Hu Bei Province}, \country{China}}}

\corres{*Hantao Zhang, Department of Computer Science, The University of Iowa, Iowa City, IA 52242, USA. \email{hantao-zhang@uiowa.edu}}

\presentaddress{Department of Computer Science, The University of Iowa, Iowa City, IA 52242, USA.}

\abstract[Abstract]{Sorting is one of the oldest computing problems and is still very important in the age of big data. 
Numerous algorithms and implementation techniques have been proposed. 
In this study, we focus on comparison based, internal sorting algorithms. 
We created 12 types of data of various sizes for experiments and tested extensively various implementations in a single setting. 
Using some effective techniques, we found that \qusort is adaptive to nearly sorted inputs and is still the best overall sorting algorithm. 
We also identified techniques that are effective in \tsortn. 
It turns out to be one of the most popular and efficient sorting \textcolor{blue}{methods} based on natural \msortn. 
In addition, we created a version of our own \msortn, which performs better than \tsort on nearly sorted instances. 
Our implementations of \qusort and \msort are different from other implementations reported in all textbooks or research articles, and are faster than any version of the C library \qsort functions not only for randomly generated data, but also for various types of nearly sorted data. 
This work can aid the user to choose the best sorting algorithm for the hard sorting job at hand, and provides a platform for anyone to test their own sorting algorithm against the best in the field.}

\keywords{Sorting algorithms; Merge sort; Quick Sort}

\maketitle


\section{Introduction}\label{sec1}

What is the best sorting algorithm?
The answer to this question is perhaps ``it depends”.  
If you are asked to recommend a sorting algorithm for an unknown problem, what is your recommendation?
According to a typical textbook on algorithms, the answer would be \emph{heapsort} or \msortn, the two asymptotically optimal algorithms, which take $\mathcal{O}(n\log(n))$ time in the worst case, where $n$ is the number of elements to be sorted. 
In practice, neither \emph{heapsort} nor \msort is a good choice. 
The most popular choice is a variant of \msort called natural \msortn. 
An equally good choice is \qusortn. 
Of course, our claim is based on experiments, not on asymptotical analysis, as nobody can do better than an optimal method. 

Sorting is arguably the most studied computing problem. 
Various algorithms and implementation techniques have been leveraged to demonstrate various algorithm design techniques.  
\textcolor{blue}{On the sorting algorithms animations website~\cite{animations}}, eight different sorting methods, that is, insertion sort, selection sort, bubble sort, shellsort, mergesort, heapsort, and two versions of \qusort (both two-way and three-way splitting), are animated on four types of initial instances: Random, Nearly Sorted, Reversed, and Few Unique. 
Their conclusion is that “there is no best sorting algorithm”, and “the initial condition (input order and key distribution) affects performance as much as the algorithm choice.” 
Apparently, this conclusion does not provide any clue to the aforementioned recommendation. 
From now on, we will use Demo \textcolor{blue}{to refer to the sorting algorithms animations website~\cite{animations}}.

In a decathlon competition, the winner is not necessarily the fastest runner or the highest jumper, but is determined by the combined performance in all events. 
If we choose the best implementation for each sorting algorithm and put them in a rigorous sort race, the winner can be decided by the combined performance in all tests. 
As in any relay competition, the fairest performance metric would be the total time taken by each method, which will also be used in our sort race. 

\begin{table}
\caption{\textcolor{blue}{Summary of implementations of the C-library \qsort function.}}
\centering
\begin{tabular}{|c|c|c|c|c|c|}
\toprule\
\# & Year & OS & Author & Algorithm\\
\midrule
1 & 1972 & Unix & Lee Mcmahon & CAR Hoare's quick sort~\cite{hoare1961algorithm}\\
2 & 1983 & BSD & Lee Mcmahon & CAR Hoare's quick sort~\cite{hoare1961algorithm}\\
3 & 1991 & BSD & Douglas Schmidt & Bentley \& McIlroy’s quick sort~\cite{bentley1993engineering}\\
4 & 1991 & GNU & Douglas Schmidt~\cite{gnuqsort} & Bentley \& McIlroy’s quick sort~\cite{bentley1993engineering}\\
5 & 1993 & GNU & Mike Haertel~\cite{gnumergesort} & John von Neumann's merge sort~\cite{knuth1998sorting}\\
\bottomrule
\end{tabular}
\label{qsortalgo}
\end{table}
To have a simple sort race, only comparison-based internal sorting algorithms will be used. 
We will call an input array of elements for sorting as an instance and use instances of at least one million elements. 
The same instance and the same comparison function will be used by all participants, and all implementations will be compiled into one single program. 
Since selection sort and bubble sort perform poorly on any large instance, they are disqualified from the race. 
To be fair, the copy of \qusort with the best performance is used.
Finally, five different sorting methods are chosen as participants in this race: insertion sort, shellsort~\cite{knuth97}, mergesort~\cite{mcilroy1993optimistic}, heapsort~\cite{williams1964algorithm}, and quicksort~\cite{hoare1961algorithm}.
%To be clear, we summarize all the \qusort algorithms used in the paper excluding our own implementations in Table~\ref{qsortalgo}. 

\begin{table}
\caption{The average running time of seven sorting methods on 4 types of inputs.}
\centering
\begin{tabular}{|c|c|c|c|c|c|c|c|c|}
\toprule
n=2000000 & best time & insert & heap & shell & merge & quick & qsort & timsort\\
\midrule
random & 225876 & -- & 3.47 & 1.73 & 1.16 & \textbf{1.00} & 1.18 & 1.15 \\
reverse sorted&4728&--&113.78&26.34&\textbf{1.00}&1.78&19.74&1.07\\
nearly sorted & 93185&1.62&5.73&1.92&\textbf{1.00} & 1.54 &1.50 &1.11 \\
few unique & 128097 & --	& 5.88 & 2.39 & 1.64 & \textbf{1.00} & 1.73 & 1.61 \\
average time & 126509 & -- &	5.18& 1.98 & 1.15 & \textbf{1.00} & 1.42 & 1.13 \\
\bottomrule
\end{tabular}
\label{table1}
\end{table}

Table~\ref{table1} shows the experimental results of our implementations of the five sorting methods plus the GNU C library function \qsort \textcolor{blue}{(Table~\ref{qsortalgo}, row 5)} and \tsortn~\cite{timsort}, on four types of \textcolor{blue}{integer lists} as used in Demo, i.e., random, reverse sorted, nearly sorted, and few unique. 
All the methods are implemented in C and compiled as a single executable by gcc with optimization “-O3”. 
The executable takes two parameters, the size and the type of the instance, and generates one instance and then run all the sorting methods on this instance. 
This way, the same set of instances is used for each method.

In Table~\ref{table1}, the column “best time” is the fastest average time (in microseconds) over 100 instances (each has 2,000,000 elements) for each of the seven methods.  
Under each method, the ratio of its running time to the best time is given. 
That is, the actual running time of that method is the multiplication of the ratio and the best time. 
If the ratio is 1.00, then the best time is produced by that method. 
The row “average time” represents the average time (in microseconds) of all 400 runs for each method, again in the format of ratio to the best time.

It is clear from the table that \qusort (nicknamed hyb2 in section 4) is the best overall method according to the best average time. 
The second best method is \msort (nicknamed mer6 in section 3), which has the best performance for reverse sorted and nearly sorted instance. 
The weak performance of $qsort$ on reverse sorted instances is typical for the standard recursive \msort algorithm on such inputs, as the \textcolor{blue}{the default GNU C-library function \qsort as of 2022 (Table~\ref{qsortalgo}, row 5) uses \emph{mergesort} but does not use natural \emph{mergesort}}.

%The performance of $qsort$, the sorting function in the C Standard Library, and timsort~\cite{timsort}, which are known as the best generic sorting methods, are given as a reference to show that our results are very competitive.  

Why the conclusion from our experiment is different from the one in Demo? 
There are two reasons: (1) size matters, and (2) implementation matters. 
%From movies, we know some people might run faster than trains, of course, in short races. 
%However, it is wrong to conclude that human are faster than trains. 
In Demo, up to 50 elements are used to test the idea of each sorting method, and this size is too small to demonstrate the strength of each method. 
For example, the overhead for choosing a good pivot in \qusort only pays off with a large size array. 
Moreover, it is very important to know that a sorting method works well or not for very large size problems. 
Every experienced programmer knows that different implementations of the same algorithm give different performances. 
For Olympics, each country sends the best athletes to compete. 
For sort race, we should choose the best implementation possible for each method.

When talking about the best sorting implementation, we cannot ignore the existence of the \qsort function in the C Standard Library.
The \qsort function has long been considered the "best" sorting function and being implemented in different libraries \textcolor{blue}{(see Table \ref{qsortalgo})}. 
Nowadays, the most popular \qsort is the GNU project’s libc implementation which powers most GNU/Linux distributions and several other operating systems. 
This \qsort is interesting in that it shuns \qusort in favor of conventional topdown \msortn, due to the poor performance of \qusort on some types of instances~\cite{gnumergesort}.  
Tim Peters created $timsort$ based on Peter McIlroy’s idea of natural mergesort~\cite{mcilroy1993optimistic}, which consists of two phases: in phase 1, sorted subarrays (called runs) are identified from the input; in phase 2, two \textcolor{blue}{runs} are merged into one run, until only one run remains. 
The two phases can be interleaved. 
There are several implementations based on natural \msortn, such as \tsortn~\cite{timsort}, and $neatsort$~\cite{la2014neatsort}. 
Different techniques are used in these implementations. 
As the first goal of this article, we will try to identify which techniques are more effective and which are less effective through extensive experiments. 
From now on, by \msortn, we mean natural \msortn~\cite{mcilroy1993optimistic}.

The BSD \qsort \textcolor{blue}{(Table \ref{qsortalgo}, row 3)} is a version of \qusort based on Bentley and McIlroy’s \qsort function. 
A significant deviation from Bentley and McIlroy’s \qsort is that the BSD \qsort resorts to insertion sort, not only for small arrays, but also whenever a partitioning round is completed without having moved any elements. 
The latter is intended to capture nearly sorted inputs, since insertion sort will handle these cases efficiently. 
However, this technique may \textcolor{blue}{missfire}. We found by accident that the BSD \qsort performs poorly on a set of instances. 
By extensive experiments, we will try to identify which techniques used in the past~\cite{sedgewick1978implementing} are effective and what new techniques can be used so that \qusort can handle better nearly sorted inputs and has better overall performances over all kinds of inputs. 
This is the second goal of this article.
The third goal of this article is to establish a software platform available in the public domain so that anyone can test their own sorting algorithm against other best implementations. 
Currently, the platform, called sortrace, is a C program which provides 12 types of inputs and over 20 implementations of sorting algorithms. 
%Each type can take another parameter to decide its subtype. 
We have tested over a dozen of different sequences in the shellsort algorithm to see which one works best. 
We hope that anyone who claims to have found a better sorting algorithm, can use this platform to sustain their claim by going through the sort race.  

%Before we select the best implementation for \qusort and \msort, let us see what types of inputs are available in the platform.     

\section{Twelve Classes of Inputs}
Twelve classes of input instances are provided on the platform.
For the first five classes of inputs, each array is randomly generated and each array element is one of the following five types: long integer, double, lists of integers of \textcolor{blue}{16, 64, and 256} bytes, respectively. 
For the other seven classes, besides the array length n, it takes another integer k ($1 \leq k \leq n$) indicating the degree of presortness.  
%Before we present these seven classes, let us first review some concepts.         

%Given an array $A = (a_1, a_2, ..., a_n)$, let rank($a_i$) be the final position of ai in the sorted array.
Many researchers prefer to use the number of inversions, \textcolor{blue}{$\mbox{\it Inv}(A)$}, 
to measure how far (or close) the array is from being sorted. If the array is already sorted, then the inversion count is 0, but if the array is sorted in the reverse order, the inversion count is the maximum which is equal to $n(n-1)/2$. Formally speaking, two elements $a_i$ and $a_j$ form an inversion if $a_i > a_j$ and $i < j$. Let \textcolor{blue}{$\mbox{\it Inv}(A)$} denote the number of inversions in the array $A$, then
%and the number of ascending runs in the list, $Run(A)$, to measure the presortedness of $A$:       
\[ \textcolor{blue}{\mbox{\it Inv}(A)} = | \{(i, j) : 1 \leq i < j \leq n, a_i > a_j\} | \]            

If we break the array $A$ at every position $i$, where $a_i > a_{i+1}$, into a collection of disjoint subarrays, then each subarray is a run, i.e., a sorted sequence of elements.  
Let $Run(A)$ be the number of such subarrays: 
\[ Run(A) = | \{i : 1 \leq i < n, a_i > a_{i+1}\} | + 1 \]     

When $A$ is sorted, $Run(A) = 1$. When $A$ is reverse sorted, the number of runs is maximal $(Run(A) = n)$; in this case, reversing the array, which takes $\mathcal{O}(n)$ time, will make $A$ sorted. 
In other words, neither \textcolor{blue}{$\mbox{\it Inv}(A)$} nor $Run(A)$ reflects the difficulty of making $A$ sorted. 
Hence, we prefer another measure over \textcolor{blue}{$\mbox{\it Inv}(A)$} and $Run(A)$. 
For instance, we are interested in the minimal number $k$ such that $A$ can be broken into $k$ subarrays, each of which is monotonic (i.e., either sorted or reverse sorted).       

\[ \mono = min\{ k : A =  w_1 w_2 … w_k, w_i  \text { is a monotonic subarray},1 \leq i \le k\} \]
           
Intuitively, if \mono\ $= k$, then $A$ is the concatenation of $k$ monotonic subarray (either sorted or reverse sorted). 
For instance, if $A$ is sorted or reverse sorted, then \mono\ $ = 1$. 
If the first half of \textcolor{blue}{$A$} is reverse sorted and the second half of $A$ is sorted, $A$ is called an “organ-pipe” array and \mono\ $ = 2$.  

Here are the five classes of inputs with an additional parameter $k$:

\begin{itemize}
\item \textbf{$k$-limited}: A simple way to create "few unique" instances is to restrict elements in a certain range. 
An array of integers is said to be \klit \ if each element $x$ in the array is a randomly generated number and $0 \leq x \leq 2k$. 
When $k = 0$, all elements of the array are the same, i.e., 0. 
When $k = 1$, the array is a randomized binary list. 
The "few unique" instance used in Demo can be mapped to a 2-limited array of 50 elements.  
If we keep the same ratio of $2^2/50 = 2^k/2,000,000$, we get $k = 12$.  
For the "few unique" instance in Table~\ref{table1}, we used 12-limited arrays.  
When $k \ge 31$, it is the same as randomized integers. 
The best sorting method for the $k$-limited is the $3$-way $quicksort$ used in Demo and is related to Dijkstra’s Dutch National Flag problem~\cite{kim2009improving} (more details can be found in section 4).

\item \textbf{$k$-equal teeth}: 
A \keq \ (or \textcolor{blue}{\emph{k-equal saw teeth}}) array can be constructed as follows: Divide an array of $n$ elements into $k$ disjoint, equal-length subarrays and each subarray is filled with the numbers from $1$ to $n/k$. Obviously, if $A$ is a \keq, $Run(A) = \mono = k$. If $k = 1$, then $A$ is sorted. 
Among all comparison-based sorting algorithms, \msort is the best one for this class of instances.

\item \textbf{$k$-even teeth}:  A \kev \ array can be constructed from \keq \ as follows: 
Reverse each odd numbered subarray of a \keq. 
When $k = 1$, the array is reverse sorted. 
For $k = 2$, it is an ``organ-pipe” array.  
Obviously, for $k \leq n/3$, if A is a $k$-even teeth, $\mono = k$. 
When $k = n/2$, the list is a 2-limited list starting with $(2,1,1,2,2,1,1,2, ...)$, and $\mono =  n/4+1$, because we may group the elements as $(2,1,1), (2,2,1,1), (2,2,1,1), ...$. 
The best sorting method for this class of instances is \msortn.

\item \textbf{$k$-sharp teeth}:
Initially, the array contains a sorted list from 1 to $n$. 
We divide the array into $k$ disjoint, equal-length subarrays and reverse the odd numbered subarrays.  
When $k =1$, the array is reverse sorted. 
If $A$ is a \ksha, $\mono = k$.  
The best sorting method for this class of instances is a special version of \msort. 
We would like to point out that the BSD $qsort$ performs very poorly on this type of inputs. 
For instance, when $n = 2,000,000$ and $k = 8$, \msort takes 0.01 sec., Bentley \& McIlroy's function takes 0.1 sec., and the BSD qsort takes 279.4 sec. 
This might explain why \qusort was shunned from the C library.

\item \textbf{$k$-shuffled teeth}: 
The array is first initialized with a \ksha, and then all $k$ subarrays are randomly shuffled by preserving the relative order of the elements in each subarray.
The concept of \kshu \ lists is related to the SMS measure in~\cite{moffat1996splaysort}, the least value $k$ such that the input list can be formed by interleaving $k$ monotone (either descending or ascending) sequences.  
Like the class of \ksha, \msort is the best for this class when $k < n/10$; when $k > n/10$, \qusort is the best. 

\item \textbf{$k$-distance}
An array $A = (a_1, a_2, ..., a_n)$ is said to be \kdis \ if $|rank(a_i) -  i| \leq k$ for all $1\leq i \leq n$, where $rank(a_i)$ is the position of $a_i$ in the array after $A$ is sorted. 
That is, for each $a_i$ in $A$, the distance from its initial position to its final position is bound by $k$. 
To create a \kdis \ list, we first construct a sorted list and then divide the list into $n/(k+1)$ subarrays, and shuffle each subarray of $k+1$ elements randomly.  
If $k = n -1$, the result is a random list. 
The "nearly sorted" instance used in Demo appears to be a 1-distance array of 50 elements.  
If we keep the same ratio of 1/50, we would use a 40000-distance array of 2,000,000 elements. 
Instead, we used 256-distance lists in Table~\ref{table1}. 
Indeed, the best performer for this class on 2,000,000 elements is insert sort if $k \leq 32$; for $k > 32$, \msort is the best.

\item \textbf{$k$-exchange}: An array $A$ is said to be a \kexc \ array if it becomes sorted after no more than exchanges of $k$ pairs elements in $A$. 
In this case,  $|\{a_i : rank(a_i) \neq i\}| \leq 2k$.  
To create a \kexc \ array, we first create a sorted list, and then repeat the following operation $k$ times: randomly choose two positions to exchange them. 
The model of \kexc \ arrays is related to the Rem measure (the minimal number of items that must be removed to leave a sorted sequence) in~\cite{moffat1996splaysort}. 
It can be shown that if $A$ is a \kexc, then \mono\ $\leq 2k+1$. 
Like the class of \kshu, \msort is the best for this class when $k \leq n$; when $k > n$, \qusort is the best.                                                
\end{itemize}              

The last six classes described above are presorted to certain degree and can be called nearly sorted with proper $k$. 
We will present experimental results of various sorting methods on these instances in later sections.  
Roughly speaking, \qusort is the best method for the first six classes of inputs: five random arrays plus $k$-limited arrays. 
\textcolor{blue}{In our experiment, we found mergesort to be} the best method for the remaining six classes of inputs, with the exception of $k$-distance arrays when $k \leq 32 (n = 2,000,000)$; in this case insertion sort is slightly faster than \msortn. 
Both \kdis \ and \kexc \ arrays model nearly sorted arrays by several researchers~\cite{cook1980best}. 
Our experimental results show clearly that \msort is the best choice for nearly sorted arrays. 
The second best is \qusortn, which performs much better than the splaysort method reported in~\cite{moffat1996splaysort}. 
Whenever we use the above 12 classes of instances, we will always include the best performers for each class so that the ratios of each method’s run time to the best time for that class will be kept the same.             

\section{Select Best Mergesort}        
The first choice for selecting the best implementation of natural \msort is perhaps \tsort~\cite{timsort}. Timsort is a highly optimized implementation of natural \msort with at least the following techniques:
\begin{enumerate}[(a)]
\item When scanning the input, if it finds a reverse sorted subsequence, it will reverse the subsequence to create a longer run. If the stable property is required, either the reverse sorted subsequence cannot contain equal elements (as done in timsort) or sub-arrays of equal elements are reversed twice (as done in our implementation).
\item If a run in the input is less than the minimal length for a run, insertion sort is called to create a longer run. 
\item The minimal length of runs can be preset, or adjusted so that if each run keeps the minimal length, then the number of runs is (or slightly less than) a power of two.
\item A binary search is used in the insertion sort to locate the position of the element to be inserted in the sorted list.
\item Runs are \textcolor{blue}{pushed} in a stack. Let $X, Y, Z$ be the sizes of the top three runs in the stack, then it maintains that $X > Y+Z$ and $Y > Z$.  
If $X < Y+Z$, then $Y$ is merged with the smaller of $X$ and $Z$ until the two conditions are satisfied. 
\item Merge is always done among two consecutive sorted subarrays, say $A$ and $B$. 
Any element in $A$ no bigger than the first element of $B$ is removed from the merge process. 
Similarly, any element of $B$ bigger than the last element of $A$ is also removed from the merge process. 
If $A$ is shorter than $B$, then $A$ is copied into a temporary memory; otherwise $B$ is copied before the merge process
\item Merge goes into the galloping mode if $A$ (or $B$) wins for a consecutive number of times (a preset number). 
In the galloping mode, the binary search is used to locate the position in A where the current element in $B$ should go.
\end{enumerate}

\textbf{Example}: Suppose the input $A = (1, 15, 18, 19, 20, 16, 11, 10, 9, 8, 7, 6, 5, 4, 3, 2)$, $Run(A) = 12$ and $\mono = 2$, because $A$ consists of one sorted subarray from 1 to 20 (a run) and one reverse sorted subarray from 16 to 2 (a reversed run).
Timsort will reverse the second subarray and merge two runs into one. 
During the merge process, the first number in $A$, i.e., 1, will be skipped because it is smaller than 2, the first number of the second run. 
Then 15 through 20 in the first run will be copied into a temporary memory. 
Once 1, 2, 3, 4 are moved into the final position, since the second run won three times in a row,  it goes into the galloping mode: using binary search, it finds that 15 falls in between 11 and 16 of the second run. 
So 5 through 11 will be moved to the final position without comparing them one by one against 15. 

For neatsort~\cite{la2014neatsort}, (a) and (f) are used; a heuristic different from (e) is used to select two runs to be merged. 
We have implemented our own \msort and have tested (a) – (g) except (d). We found that (a), (f) and (g) have great impact to the performance of \msortn, so all of our implementations will use them. 
We also found that (b) provides a little performance improvement for most cases. 
However, we found that (c) has almost no impact. 
The motivation behind (c) is to avoid the merge of a very long run with a very short run. 
If the sizes of each run cannot be the same, the number of runs can hardly be an expected power of two. 
For (d), we found that using binary search in insertion sort, the impact to the running time is marginal, but it reduces the number of comparisons substantially. 
When expensive comparison functions are used, we still do not see the gain on running time. Instead of (e), we have tested two new strategies to select runs to merge:
\begin{enumerate}[(e$_1$)]
\item Merge of runs takes multiple rounds: in each round, all runs are paired and merged into one.
\item Maintain a list of runs such that the length of each run is less than the half of the length of its predecessor. If the condition is not true, merge this run with its predecessor. Such a list is called “half-down list”.
\end{enumerate}

It turns out that (e), (e$_1$), (e$_2$) or $neatsort$’s heuristic have a small impact \textcolor{blue}{on} the overall performance.

\begin{table}
\caption{Performances of 8 different implementations of \msort for 12 classes of inputs.}
\centering
\begin{tabular}{|c|c|c|c|c|c|c|c|c|c|}
\toprule
n=2000000 & best  & qsort & mer2 & mer3 & mer4 & mer5 & mer6 & Tim & neat\\
\midrule
random 16-list	&764548&	1.43&	1.74&	1.80&	1.77&	1.32&	1.27&	1.21&	1.80\\
random 64-list&	1279070&	1.25&	1.84&	1.99&	1.89&	1.26&	1.25&	1.19&	2.00\\
random 256-list&	1530270&	1.24&	1.81&	1.92&	1.83&	1.19&	1.20&	1.16&	1.88\\
random double&	249758&	1.14&	1.07&	1.12&	1.08&	1.07&	1.01&	1.11&	1.15\\
random integer&	231084&	1.17&	1.11&	1.16&	1.12&	1.12&	1.06&	1.13&	1.17\\
k-limited&	142186&	1.53&	1.30&	1.39&	1.30&	1.37&	1.26&	1.35&	1.51\\
k-equal teeth&	25180&	3.24&	1.38&	1.03&	1.05&	\textbf{1.00}&	1.03&	1.10&	1.20\\
k-even teeth&	24712&	3.56&	2.90&	1.04&	1.19&	\textbf{1.00}&	1.14&	1.11&	1.21\\
k-sharp teeth&	13846&	6.14&	4.47&	1.15&	1.34&	\textbf{1.00}&	1.26&	1.22&	1.61\\
k-shuffled teeth&	96920&	1.54&	1.04&	1.14&	\textbf{1.00}&	1.16&	1.01&	1.07&	1.49\\
k-distance&	51051&	2.15&	\textbf{1.00}&	1.43&	1.05&	1.49&	1.10&	1.27&	1.38\\
k-exchange&	10075&	8.55&	1.78&	1.21&	1.24&	\textbf{1.00}&	1.09&	1.11&	3.28\\
\textbf{Average  (12)}&	400872&	1.24&	1.56&	1.64&	1.57&	1.11&	1.09&	1.08&	1.65\\
\textbf{Average  (8)}&	82566&	1.65&	1.18&	1.10&	1.03&	1.08&	\textbf{1.00}&	1.07&	1.23\\

\bottomrule
\end{tabular}
\label{table2}
\end{table}

Table~\ref{table2} provides the performance results of eight different implementations of \msort for 12 classes of inputs. The value of $k$ is 2$^i$, where $i = 0, 1, ..., 8$, for the last seven classes of inputs. The column “best” shows the best average running time of 100 executions, in microseconds. It may be produced by other implementations not listed in the table. The running times of each method are given as a ratio to the best average time. If it is 1.00, then this is the best performer among all sorting methods in our experiment. The column “qsort” gives the running time of the GNU qsort. The columns mer2–mer6 give the running times of our five versions of \msort, which differ by the used techniques:
\begin{itemize}
\item mer2: uses (b), (f), (g), (e$_1$)
\item mer3: uses (a), (f), (g), (e$_1$)
\item mer4: uses (a), (b), (f), (g), (e$_1$)
\item mer5: uses (a), (f), (g), (e$_2$)
\item mer6: uses (a), (b), (f), (g), (e$_2$)
\end{itemize}

In fact, mer2 is not natural merge sort, as runs are not discovered from the input but created by insertion sort. 
The code of mer6 is given in the Appendix~\ref{ccode} for reference.  
The last row gives the average running times of the last 8 classes of inputs (not the average of ratios). 
The second last row gives the average time of all 12 classes. 
Mer6 is the best performer of the last 8 classes of inputs; mer4 is the second best (3\% slower) and timsort is the third (7\% slower). 
The GNU qsort is the worst performer in the table (65\% slower).

When all the 12 classes are considered, \tsort is the best among all \msort implementations in our experiment, and is only 7\% slower than the best \qusort implementation. Mer6 is the second best (1\% slower than \tsortn). 
The advantage of \tsort over mer6 is shown on the first three randomized classes where expensive comparison functions are used; for random integers or doubles, mer6 is still faster than \tsort. 
The idea of (d) cannot reduce the running time of mer6 while it can reduce the number of comparisons. 
This may suggest that (e) is better than (e$_1$) and (e$_2$) for large size lists with high comparison costs. 

For the three contrived classes, \keq, \kev, and \ksha, mer5 is the best performer among all implementations, while the GNU \qsort\textcolor{blue}{(Table \ref{qsortalgo}, row 5)} is at least three times slower than mer5. 
For k-exchange, \qsort is almost eight times slower than mer5.  

\section{Select Best QuickSort}       
As the most popular sorting algorithm, \qusort has been extensively studied by many researchers and various techniques have been proposed. 
In this study, we consider Bentley \& McIlory(B\&M)’s \qsort implementation \textcolor{blue}{(Table \ref{qsortalgo}, row 3)}. 
Let $n$ be the input array size, the following techniques are in B\&M \qsort function: 

\begin{enumerate}[(a)]
\item If $n < \alpha$, call insertion sort. Bentley and McIlroy use $\alpha = 7$.
\item If $n < \beta$, use the median of the first, middle and the last elements as pivot for splitting the array. Bentley and McIlory use $\beta = 40$.
\item If $n \geq \beta$, use the pseudo-median of 9 elements (taken at every $(n/8)^{th}$) position plus the first element) as pivot.
\item During the splitting process, all elements equal to the pivot are placed either at the beginning or end of the array. When the splitting is done, these elements equal to the pivot are moved in the middle of the array and are spared from the recursive calls.
\item Do not identify the position of the pivot in the array. Bentley and McIlroy use this idea for all eight byte elements.
\end{enumerate}

The use of (c) is to guarantee that a good pivot element is chosen, to avoid the quadratic worst-case time complexity.  
The use of (a) is to avoid the overhead of choosing a better pivot. The idea behind (d) is called 3-way splitting, and \qusort using (d) is called the 3-way \qusortn. 
Evidently, when no repeated elements are present, 3-way splitting is slower than the classic 2-way splitting~\cite{10.5555/1614191}. 
The idea of (e) is to avoid swapping the pivot with other elements so that the presortedness of the input is not disturbed. 

For the 3-way splitting, at the end of splitting, the array is divided into three parts: the first part contains elements less than the pivot; the middle part contains elements equal to the pivot and the last part contains elements greater than the pivot. 
Recursive calls apply to the first and the last parts. 
The 2-way splitting in any standard textbook~\cite{10.5555/1614191} also splits the array into three parts: the elements less than the pivot, the pivot, and the elements no less than the pivot. 
Using (e) without (d), the array is split into two parts: the elements no greater than the pivot and the elements no less than the pivot. 

%The BSD \qsort function \textcolor{blue}{(Table \ref{qsortalgo}, row 3)} in the C library has some deviation from \textcolor{blue}{B\&M \qsortn}: the BSD \qsort also \textcolor{blue}{uses} insertion sort instead of recursive calls of \qsort, whenever a splitting round is completed without having moved any elements. 
%This idea is intended to capture nearly sorted inputs and may perform poorly for certain inputs. 
%For instance, if an array contains $2m+1$ distinct integers, the first $m$ integers are the reversed list of numbers, the middle element is $m+1$, the last $m$ integers are the reversed $m+2$ through $2m+1$, then the BSD \qsort will use insertion sort to sort two reversed lists of $m$ elements. 
%We have tested the BSD \qsort on the 12 classes of inputs and its performance is very poor for the class of $k$-sharp teeth arrays. 
Besides the ideas used by Bentley and McIlroy, we also have tested the following two ideas:

\begin{enumerate}[(f)]
\item If $n \geq \alpha$, test if the array is already sorted; if yes, exit.
\end{enumerate}
\begin{enumerate}[(g)]
\item At first, use the 2-way splitting. If the array contains more than $\gamma = 2$ elements equal to the pivot, use the 3-way splitting in the recursive calls.
\end{enumerate}

The use of (f) is to avoid wasting time on already sorted subarrays, which occur often in nearly sorted inputs. 
The use of (g) is a typical example of hybrid algorithms.  
The program starts with the 2-way \qusortn; when it identifies repeated elements, it turns to the 3-way \qusortn. 
 \textcolor{blue}{Combining} (e) and (f), reverse sorted lists are no longer a problem for \qusort: Using the median of the input as pivot, the splitting will reverse the list and the two recursive calls will finish quickly with the use of (f). 
In other words, \qusort becomes adaptive as it takes O(n) time on sorted or reverse sorted lists. 
It also works well for some other troublesome instances including organ-pipe lists.

\begin{table}
\caption{Performance of \qusort and \msort implementations for 12 classes of inputs.}
\centering
\begin{tabular}{|c|c|c|c|c|c|c|c|c|c|c|c|c|}
\toprule
n=2000000 & best  & B\&M & 3-way & 2-way & hyb1 & hyb2 & hyb3 & hyb4 & mer5& mer6 & Tim & qsort \\
\midrule
random 16-list	&764548	&1.11&	1.12&	1.28&	1.10&	\textbf{1.00}&	1.08&	1.00&	1.32&	1.27&	1.21&	1.43\\
random 64-list&	1279070	&1.06	&1.05	&1.02	&1.03	&1.02	&1.03	&\textbf{1.00}	&1.26	&1.25	&1.19	&1.25\\
random 256-list&	1530270&	1.13&	1.09&	1.04&	1.04&	\textcolor{blue}{\textbf{1.00}}&	1.04&	1.03&	1.19&	1.20&	1.14&	1.24\\
random double&	249758	&1.17	&1.13	&1.05	&1.04	&1.05	&\textbf{1.00}	&1.06	&1.07	&1.01	&1.11	&1.14\\
random integer&	231084&	1.18&	1.11&	1.03&	1.03&	1.05&	\textbf{1.00}&	1.05&	1.12&	1.06&	1.13&	1.17\\
k-limited&	142186	&1.13	&1.07	&1.10	&1.03	&1.04	&\textbf{1.00}	&1.04	&1.37	&1.26	&1.35	&1.53\\
k-equal teeth&	25180&	5.54&	4.71&	4.93&	4.67&	4.36&	4.17&	4.66&	\textbf{1.00}&	1.03&	1.10&	3.24\\
k-even teeth&	24712	&5.72	&4.84	&4.97	&4.82	&4.52	&4.28	&4.81	&\textbf{1.00}	&1.14	&1.11&	3.56\\
k-sharp teeth&	13846	&6.14&	2.42&	1.94&	2.28&	2.12&	1.85&	2.04&	\textbf{1.00}&	1.26&	1.22&	6.14\\
k-shuffled teeth&	96920	&2.37	&2.09	&1.95	&1.96	&1.97	&1.88	&1.98	&1.16	&1.01	&1.07	&1.54\\
k-distance&	51051	&2.53&	2.41&	2.04&	2.09&	2.17&	1.85&	2.21&	1.49&	1.10&	1.27&	2.15\\
k-exchange&	10075	&9.52	&2.95	&2.44	&2.69	&2.69	&2.18	&2.46	&\textbf{1.00}	&1.09	&1.11	&8.55\\
\textbf{Average  (12)}&	400872&	1.23&	1.06&	1.04&	1.02&	\textbf{1.00}&	1.01&	1.01&	1.11&	1.09&	1.08&	1.24\\
\textbf{Average  (8)}&	82566	&1.93	&1.56	&1.48	&1.47	&1.46	&1.37	&1.49	&1.08	&\textbf{1.00}	&1.07	&1.65\\
\bottomrule
\end{tabular}
\label{table3}
\end{table}

Table~\ref{table3} shows the performance of seven implementations of \qusort on the same set of inputs used in Table~\ref{table2}. 
For reference, we also copied four versions of \msort from Table 2: mer5, mer6, Tim and \qsortn. 
The seven quicksort implementations are: (1) B\&M, Bentley and McIlroy’s \qsort function. (2) 3-way: the 3-way \qusort uses the ideas of (a) to (f), and is almost identical to B\&M except (f), i.e., test if the array is sorted as a preprocessing step. (3) 2-way: the 2-way \qusort using the ideas of (a), (b), (c), (e), and (f). (4)-(7): Four variants of hybrid \qusort using the ideas of (a)-(g), i.e., hyb1, hyb2, hyb3, and hyb4, with $(\alpha, \beta) = (16, 16), (32, 32), (32, 64),$ and $(16, 32)$ respectively. 
The BSD \qsort is excluded here because it performs poorly on the k-sharp teeth instances when $k \geq 5$.

The performances of all methods on randomized inputs (the first five classes) are about the same, with the GNU \qsort \textcolor{blue}{(Table \ref{qsortalgo}, row 5)} being the worst performer. 
The high cost on the class of random 256-lists is due to an expensive comparison function and high number of cache misses. 
Comparing to \msort, \qusort as a \textcolor{blue}{family} is slower on non-randomized inputs. 
%However, these inputs are no worse than randomized inputs. 
In fact, the hybrid \qusort is the best performer over all the 12 classes of inputs. 
If we have to select \textcolor{blue}{one implementation} of \qusortn, then hyb2 $(\alpha = \beta = 32)$ is the choice according to Table~\ref{table3}. 
The GNU \qsort \textcolor{blue}{(Table \ref{qsortalgo}, row 5)} and \textcolor{blue}{B\&M \qsort (Table \ref{qsortalgo}, row 4)} are about 23\% slower than hyb2. 
The use of idea (f) is indispensible for this performance as it helps \qusort to save time on nearly sorted inputs. 

Sarwar et al.~\cite{sarwar1996engineering} once conducted \textcolor{blue}{extensive experiments} on \qusort and tried various methods for selecting a pivot (from 1, 3, 5, 9, and 17 elements). 
They concluded that \qusort performs best when the pivot is the median of three elements and there is no need to call insertion sort. 
Our experiment shows that their claims are true only for randomized inputs. 
For 2-equal teeth or 2-even teeth, which are called the organ-pipe shape inputs, if we let $\beta = \infty$, \qusort will always pick the median of three elements as pivot and in this case, it exhibits the $\mathcal{O}(n^2)$ worst-case behavior. 
The idea of (c), i.e., use the pseudo-median of 9 elements as pivot, essentially prevents the worse-case from happening. 
In this case, the idea (a) helps to reduce the overhead of (c). When $\alpha = \beta$, (b) is excluded and (c) becomes the only strategy for selecting a pivot. 
Our experiment shows that as long as $16 \leq \alpha \leq \beta \leq 100$, the performance of \qusort has little difference on random inputs.

\section{Count Numbers of Comparisons} 
In the previous sections, we used the running time as the metric to select the best performer. 
In the GNU C library \qsort function, the comparison function must be passed as a parameter to \qsort and we use the interface of \qsort as a model for every sorting method. 
Since we have control over the comparison function, we can count how many comparisons a sorting algorithm performs, which is also a pretty good metric.

%In the previous sections, we used the running time as the metric to select the best performer. 
%In the C library \qsort function, the order and how to compare elements is defined by a comparison function which the caller must pass to \qsortn(). 
%Since we control the comparison function and we use the interface of \qsortn() as a model for every sorting method, we can also count how many comparisons an implementation performs, which is also a pretty good metric.

\begin{table}
\caption{Average number of comparisons per element (n = 2,000,000).}
\centering
\begin{tabular}{|c|c|c|c|c|c|c|c|c|c|c|c|c|}
\toprule
cmp per item & 3-way & 2-way & hyb2 & hyb4 & B\&M& qsort & mer4 & mer5 & \textcolor{blue}{mer6}  & \textcolor{blue}{Tim}\\
\midrule
random 16-list	&22.51	&24.62&	24.47&	23.47&	22.57&	19.68&	21.62&	20.77&	21.81&	19.20\\
random 64-list&	22.74	&24.73	&24.70	&23.41	&22.08	&19.67	&21.62	&21.09	&22.22	&19.64\\
random 256-list&	22.92&	24.83&	24.80&	23.50&	22.07&	19.67&	21.62&	21.04&	22.17&	19.64\\
random double&	22.79	&24.83	&24.79	&23.51	&22.08	&19.67	&21.62	&21.09	&22.16	&19.64\\
random integer&	22.95	&24.80	&24.77	&23.48	&22.13	&19.67	&21.62	&21.03	&22.15	&19.64\\
k-limited&	13.65	&16.64&	14.63&	14.06&	13.30&	18.89&	15.89&	15.23&	16.17&	14.19\\
k-equal teeth&	19.24	&22.86	&19.68	&19.24	&20.59	&11.23	&4.02	&4.02	&4.02	&4.02\\
k-even teeth&	18.78&	22.51&	19.36&	18.91&	20.44&	11.31&	4.67&	4.05&	4.49&	4.16\\
k-sharp teeth&	6.65	&6.39	&6.46	&6.58	&19.90	&9.32	&1.33	&0.89	&1.33	&0.89\\
k-shuffled teeth&	19.07&	20.54&	20.48&	19.49&	21.61&	16.57&	8.52&	7.18&	8.43&	6.71\\
k-distance&	20.26	&20.62	&20.53	&20.85	&20.63	&11.72	&5.40	&5.31	&5.35&	4.99\\
k-exchange&	6.07&	6.07&	6.07&	6.07&	19.69&	12.36&	1.00&	1.00&	1.00&	1.00\\
\textbf{Average  cmp}&	17.64	&19.70	&18.56	&17.96	&20.09	&15.81	&12.41	&11.89	&12.61	&11.14\\
\bottomrule
\end{tabular}
\label{table4}
\end{table}

Table~\ref{table4} lists the average number of comparisons per item for the same inputs in Table~\ref{table3} $(n = 2,000,000)$ for the selected 10 sorting methods, the first five from \qusort, and the second five from \msortn. 
That is, each number is the average of 100 numbers of comparisons per item for randomized arrays. 
To compute the average number of comparisons for each $k$-class, we chose 9 values for $k$. 
The value of $k$ is $2^i$, where $i = 0, 1, …, 8$. 
For each class, we took the average number of comparisons from 50 executions and then took the average for the whole class. 
The use of binary search in insertion sort can reduce the number of comparisons by about 10\% (not shown here) for those methods using insertion sort.

For $n = 2,000,000$, hyb4 makes on average $19n$ comparisons for \keq \ and \kev \ arrays, lower than $23n$ comparisons for randomized lists (and much lower than $38n$ of heapsort for \keq\ and \kev\ arrays). 
The idea of testing whether an array is already sorted is used in the first four \qusort methods. 
However, Table~\ref{table4} does not show any significant increases in the number of comparisons. 
In fact, the small values for the \ksha\ and \kexc\ classes tell that this idea is very useful.

As an approximation, we may assume that the number of comparisons is defined by a function $f(n) = sn\log(n) + tn$ on an input array of $n$ elements, where $s$ and $t$ are constants. 
The number of comparisons per item is $g(n) = f(n)/n = s\log(n) + t$, which happens to be a line on $\log(n)$. 
To determine the values of $s$ and $t$, we run our implementations on the 12 classes of inputs of various sizes: there are five values for $n = j10^6$, where $j = 1, ..., 5$. 
We then used linear regression to compute the approximate values of $s$ and $t$ (each line is computed over 5 points).

\begin{table}
\caption{Value of \textcolor{blue}{$s$} in the number of comparisons function $f(n) = sn\log(n) + tn$.}
\centering
\begin{tabular}{|c|c|c|c|c|c|c|c|c|c|c|c|c|}
\toprule
value of s & 3-way & 2-way & hyb2 & hyb4 & B\&M& qsort & mer4 & mer5 & mer6 & Tim \\
\midrule
random 16-list	&1.03&	1.06&	1.01&	1.04&	1.10&	1.00&	1.13&	1.02&	1.01&	1.02\\
random 64-list&1.14	&1.11	&1.11	&1.11	&1.13	&1.01	&1.13	&1.02	&1.01	&1.02\\
random 256-list&	1.09&	1.07&	1.07&	1.07&	1.07&	1.01&	1.13&	1.03&	1.05&	1.02\\
random double&	1.07	&1.02	&1.01	&1.02	&1.09	&1.01	&1.13	&1.01	&1.02	&1.02\\
random integer&	1.06	&1.12&	1.12&	1.12&	1.06&	1.01&	1.13&	1.04&	1.05&	1.02\\
k-limited&	0.46	&0.50	&0.46	&0.48	&0.48	&0.96	&0.54	&0.49	&0.50	&0.51\\
k-equal teeth&	1.24&	1.27&	1.23&	1.23&	1.31&	0.41&	0.01&	0.01&	0.01&	0.00\\
k-even teeth&	1.22	&1.21	&1.27	&1.27	&1.31	&0.45	&0.00	&0.00	&0.00	&0.00\\
k-sharp teeth&	0.04	&0.03&	0.03&	0.02&	0.98&	0.45&	0.00&	0.00	&0.00	&0.00\\
k-shuffled teeth&	0.79	&0.77	&0.78	&0.79	&1.07	&0.84	&0.00	&0.00	&0.00	&0.00\\
k-distance&	1.11	&1.10&	1.10&	1.08&	1.02&	0.49&	0.00&	0.00&	0.00&	0.00\\
k-exchange&	0.02	&0.02	&0.02	&0.02	&1.06	&0.45	&0.00	&0.00	&0.00	&0.00\\
\bottomrule
\end{tabular}
\label{table5}
\end{table}

\begin{table}
\caption{Value of $t$ in the number of comparisons function $f(n) = sn\log(n) + tn$.}
\centering
\begin{tabular}{|c|c|c|c|c|c|c|c|c|c|c|c|c|}
\toprule
value of $t$ & 3-way & 2-way & hyb2 & hyb4 & B\&M& qsort & mer4 & mer5 & mer6 & Tim \\
\midrule
random 16-list	&-1.03	&1.36	&1.31	&0.56	&-0.59	&-1.34	&-1.53	&-0.43	&0.16&	-1.80\\
random 64-list&-1.09&	1.47&	1.44&	0.13&	-1.53&	-1.40&	-1.98&	-0.38&	1.08&	-1.77\\
random 256-list&0.12	&2.40	&2.38	&1.09	&-0.34	&-1.39	&-1.97	&-0.53	&0.22	&-1.77\\
random double&0.47&	 3.47&	 3.69& 	2.16&	-0.76	 &-1.38&	-1.94&	-0.11&	0.74&	-1.81\\
random integer&0.83& 	1.33&	1.40&	0.01&	-0.15&	-1.39&	-1.95&	-0.80&	0.13&	-1.79\\
k-limited&	3.93	&6.27	&4.93	&4.00	&3.29	&-1.13	&4.57	&5.05	&5.66	&3.49\\
k-equal teeth&	-6.66	& -3.77&	-6.03&	-6.46&	-6.93&	2.72&	3.80&	3.80&	3.80&	4.02\\
k-even teeth&	-6.85	& -2.83	 &-7.33	&-7.77	&-7.06	&1.86	&4.67	&4.05	&4.49	&4.16\\
k-sharp teeth&	5.73	&5.70&	5.75&	6.90&	-0.55&	-0.09&	3.36&	3.89&	3.35&	4.91\\
k-shuffled teeth&2.53	&4.32	&4.17	&3.05	&-0.71	&-0.97	&8.48	&5.43	&5.55	&6.68\\
k-distance&-2.91&	-2.46&	-2.47&	-1.67&	-0.68&	1.42&	5.73&	5.32&	5.36&	5.53\\
k-exchange&	5.65	&5.65	&5.65	&5.65	&-2.41	&2.92	&1.05	&1.05	&1.05	&1.05\\
\bottomrule
\end{tabular}
\label{table6}
\end{table}

Tables~\ref{table5} and~\ref{table6} give the values of $s$ and $t$ for the selected 10 sorting methods. 
For randomized inputs, the values of $s$ for the 10 methods are close to 1, indicating these are very good sorting methods. 
For instance, for hyb4 over random doubles, the number of comparisons is $f(n) = 1.02n\log(n) + 2.16n$. 
These numbers also indicate that the relative strength of these method tend to be the same as comparison functions of different costs are used. 
For the classes of \keq\ and \kev\ arrays, the $s$ values of \qusort are slightly larger than 1, indicating that these are hard problems for \qusort. 
For the last six classes, with the exception of the GNU \qsort \textcolor{blue}{(Table~\ref{qsortalgo}, row 5)}, the $s$ values of the \msort methods are (almost) zero. 
That is, these methods take linear time to sort these six classes.  
That explains why these methods are so fast on these inputs.

\section{Conclusions}
Through extensive experiments, we have achieved our goals for this project. 
\begin{itemize}
\item We have suggested 12 classes of inputs which exhibit different degree of presortedness and can be easily generated by a program. 
Using them, we can identify the best sorting methods and test other methods against them. 
For instance, we have tested a dozen of series for shellsort~\cite{knuth97} and found that the best shellsort is still 180\% slower than the best mergesort or quicksort over the 12 classes of inputs. 
The biggest disappointment is heapsort, which has the optimal complexity of $\mathcal{O}(n\log(n))$, but performs poorly. 
For the bottom-up heapsort~\cite{williams1964algorithm}, which is claimed to be much faster than \qusortn~\cite{wegener1993bottom}, it is about three times slower over the 12 classes of inputs and 7 times slower over the six classes of nearly sorted inputs. 
Smoothsort~\cite{dijkstra1982smoothsort}, a variant of heapsort, runs about two times slower than \qusortn. 
Splaysort~\cite{moffat1996splaysort} is also much slower than \msort and \qusort on nearly sorted inputs. 
All the sorting methods  mentioned above and in the previous sections are implemented in a C program called \textit{sortrace}. 
The 12 inputs can also be generated from the same program. 
That is the software platform we plan to extend, as we will add more types of inputs into the platform. 
The C code of \textit{sortrace} is publicly available~\cite{sortracecode}.
\item For sorting methods based on natural \msortn~\cite{mcilroy1993optimistic}, we have identified three techniques of \tsortn~\cite{timsort} as important: (1) reversing revered runs; (2) the galloping merge, and (3) management of runs to decide which and when to merge two runs. 
The weakness of neatsort~\cite{la2014neatsort} is perhaps due to the missing of (2). 
We proposed a new way to manage runs (in mer5 and mer6 using the half-down list) which has better performance than \tsort for nearly sorted inputs. 
The use of insertion sort is helpful as evident by the slight gain of mer6 over mer5. 
The use of binary search in insertion sort has high impact on the number of comparisons but little impact on running time. 
The adjustment of the minimal run length is easy to implement but does not have any significant impact on performance.
\item For sorting methods based on \qusortn, we confirmed that Bentley and McIlroy’s method of choosing a pseudo-median over 9 elements as pivot is crucial to avoid the worst case complexity of $\mathcal{O}(n^2)$. 
We proposed two new ideas to improve \qusortn: (1) testing sortedness as a preprocessing step in each recursive call; (2) a hybrid \qusort combing the 2-way \qusort with the 3-way \qusortn. The idea of (1) is to make \qusort adaptive to nearly sorted inputs. The idea of (2) is to avoid the overhead of 3-way splitting when there are few identical elements. As a result, our \qusort is the overall champion for the 12 classes of inputs. For the 8 k-classes of inputs, our \qusort also beats the GNU \qsortn, even though the worst-case time complexity of \qusort is worse than that of \msort (or heapsort). For the textbook version of the \qusort algorithm~\cite{10.5555/1614191}, the worst-case time complexity comes when the input list is already sorted or near-sorted. This property was regarded as a significant drawback of \qusortn. Our experiment shows that using the proposed techniques in this article, we can not only avoid the worst-case complexity completely but also make \qusort adaptive to nearly sorted inputs. Apparently, the C library should not shun \qusortn.
\end{itemize}

We have tried to avoid theoretic analysis and technical details in this article. 
As further research, we are interested in a formal analysis of the adaptiveness of \qusort when sortedness testing is used as a preprocessing step. 
Apparently such testing uses more comparisons.  
However, the experiment shows no significant increase in the number of comparisons. 
For instance, for hyb4 on $n$ randomized doubles, the number of comparisons is \textcolor{blue}{$f(n) = 1.02n\log(n) + 2.16n$}, while Bentley and McIlroy’s qsort uses \textcolor{blue}{$f(n) = 1.09n\log(n) + 0.76n$}.  
Moreover, we would like to see how the complexity of \msort and \qusort is related to the metric \mono.

One advantage of \msort over \qusort is that \msort is stable. In general, \msort takes $\mathcal{O}(n)$ more working memory than \qusort as required by merging. 
In our experiments, we see that both methods slow down on long lists of heavy items. Timsort works better than our \msort on such case, but is still slower than \qusortn. 
\textcolor{blue}{This is evident from Table~\ref{table3} that hyb2 is shown to be the best for lists of 256 numbers.}
Our experiment has been run on a linux machine with an Intel(R) Xeon(R) CPU E5-2667 (20M Cache and 3.20GHz) and 64GB memory. 
Further experiments are needed to test the limits of both methods and search for better techniques to deal with memory shortage. 
We would also like to test our program on different machines and investigate the impact of cache memory. 

Regarding the question raised in the beginning of this article, “if you are asked to recommend a sorting algorithm for an unknown problem, what is your recommendation?”  What will be your answer after reading this article?  Our answer is \qusortn, especially our hybrid \qusortn, if stable sorting is not required. 
When stable sorting is required, or the input contains some kind of presortedness and the memory is not a concern, then \msortn, either \tsort or ours. 
\textcolor{blue}{When your task needs to sort short lists ($n \le 2000$) many times, our recommendation is still \qusortn, either 3-way or hybrid, as the top 3 performers all come from the \qusort family for small lists in our experiment.}

\section{Acknowledgements} 
Special thanks go to Marcello La Rocca who converted their neatsort into C and gave it to us. 
The C codes of timsort, smoothsort, bottom-up heapsort, and splaysort are from the following websites: 
\begin{itemize}
\item \url{https://github.com/patperry/timsort};
\item \url{https://en.wikibooks.org/wiki/Algorithm_Implementation/Sorting/Smoothsort};
\item \url{https://github.com/Maxime2/heapsort/blob/master/dps\_bottom\_up\_heapsort.c};
\item \url{http://people.eng.unimelb.edu.au/ammoffat/splaysort}.
\end{itemize}
We thank the authors of these codes for making them available. 
\textcolor{blue}{We also thank the anonymous reviewers for their careful reading of our manuscript and their insightful comments and constructive suggestions}.

\bibliography{ref-AMA}

\appendix
\section{Our implementation of merge sort mer6 in C}
\label{ccode}
\begin{lstlisting}[style=CStyle]
// merge(arr, left, middle, right) merges arr[left, middle-1] and arr[middle, right-1] into arr[left, right-1]
// reverse(arr, left, right) reverse the subarray arr[left, right].

void mergesort6 (WORD *arr, size_t n, size_t es, int (*cmp)(const void *, const void *)) {                                                                                       
    int *run, next, j, s, t, runnum, start;                                                                                                                                             
    run = (int *) malloc(64*sizeof(int));  // size of run is log(n) - 64 is more than enough.
    runnum = run[0] = start = 0; next=1;                                                                                                                                                                        
    while (next < n) {                                                                                                                                                             
         if (cmp(arr+next-1, arr+next)>0) {   // i.e., if (arr[next-1] > arr[next])     
                                                                                                                                      
           if (start+minrun > next) {     // make sure each run has at least minrun elements                                                                                                        
                j = next+1; s = 0;  // look for reversed run and reverse equal elements                                                                                                                                                              
                while (j < n && ((t=cmp(arr+j-1, arr+j))>=0)) {                                                                                                                       
                   if (t == 0) s++;                                                                                                                                                    
                   else if (s > 0) { reverse(arr, j-1-s, j-1);  s = 0;  }   // i.e., reverse arr[j-1-s .. j-1].                                                                                                                                                       
                   j++;                                                                                                                                                                
                }                                                                                                                                                                      
                if (next > start+1) {
                     reverse(arr, next, j-1);              // reverse from next to j-1, then merge two parts.                                                                                                                                          
                     merge(arr, start, next, j);  
                } else  { reverse(arr, start, j-1); }   // reverse from start to j-1                                                                                                                                                                                                                                                         
                next = j;   

                if (start+minrun > next)  {      // still too short, use insertion sort to extend the run                                                                                                       
                    j = (start+minrun<n)? start+minrun : n;                                                                                                                                               
                    isort(arr, start, next, j);      // insertion sort arr[start..j-1] where arr[next..next-1] is sorted.                                                                                                                    
                    next = j;                                                                                                                                                                                      
                } else { // of (start+minrun > next)                                                                                                                                       
                    // record a new run                                                                                                                                                   
                    run[++runnum] = start = next++;                                                                                                                                                       
                                                                                                                                                                                
                    // merge two runs if the current  run > 1/2 of its predecessor                                                                                                         
                    while (runnum > 1 && (run[runnum]-run[runnum-1]) > (run[runnum-1]-run[runnum-2])>>1) {                                                                                                         
                         j = run[runnum-2];                                                                                                                                                       
                         merge(arr, j, run[runnum-1], run[runnum]);                                                                                                                             
                         run[runnum-1] = run[runnum];                                                                                                                                                  
                         runnum--;                                                                                                                                                                
                    }                                                                                                                                                                     
                }                                                                                                                                                                       
            } else next++;                                                                                                                                                               
       }                                                                                                                                                                           
                                                                                                                                                                                
       // merge all runs into one, from last to first.                                                                                                                             
       while (runnum > 0) { 
             j = run[runnum-1]; 
            merge(arr, j, run[runnum--], n); 
       }                                                                                                           
       free(run);                                                                                                                                                                  
  }                                                                                                                                                                                
\end{lstlisting}

\begin{comment}
\section{The java code of hybrid quicksort}\label{javacode}                
To better present our ideas, we give below the Java code of our algorithm, while the experiment is conducted using a C implementation.
\begin{lstlisting}
public void qsort_hyb(Item[] arr, int low, int high) {                                                                                                                                                                                                                                                 
    Item v;                                                                                                                                                  
    int i, j, r, c=0, n=high-low+1;                                                                                                                                              
                                                                                                                                                                   
    if (n < alpha) { isort(arr, low, high); return; }                                                                                                                     
    if (is_sorted(arr, low, high)) return;                                                                                                                              
    if (n < beta) v = pick_pivot3(arr, low, high);          // v is the pivot                                                                                 
    else v = pick_pivot9(arr, low, high);                   // v is the pivot                                                                                 
          
    i = low; j = high;
    while (true) {                                                                                                                                               
        while ((r=v.compareTo(arr[i])) < 0) i++;  if (r==0) c++;                                                                                                                                         
        while ((r=v.compareTo(arr[j])) > 0) j--;  if (r==0) c++;                                                                                                                                         
        if (i >= j) break;                                                                                                                                   
        exchange(arr, i++, j--);                                                                                                                                                                                                                                                               
    }                                                                                                                                                        
                                                                                                                                                                   
    if (c > 2) {                                                                                                                                             
    	if (low < j) qsort_3way(arr, low, j);                                                                                                           
       if (i < high) qsort_3way(arr, i, high);                                                                                                                      
    } else {                                                                                                                                                 
       if (low < j) qsort_hyb(arr, low, j);                                                                                                           
       if (i < high) qsort_hyb(arr, i, high);                                                                                                              
    }                                                                                                                                                        
 }           
\end{lstlisting}

The following methods are used in the method qsort\_hyp: 
\begin{lstlisting}
// insert sort on subarray arr[low..high]:
public void isort(Item[] arr, int low, int high) { }
// return true iff arr[low..high] is sorted:        
public boolean is_sorted(Item[] arr, int low, int high) { }
// return the median of arr[low], arr[(low+high)/2], and arr[high]:
public Item pick_pivot3(Item[] arr, int low, int high) { }
// return the pseudo-median of nine elements:
public Item pick_pivot9(Item[] arr, int low, int high) { }
// exchange elements at positions i and j in arr:
public void exchange(Item[] arr, int i, int j) { }
// an implementation of 3-way quicksort (qsort_3way is used in the recursive calls): 
public void qsort_3way(Item[] arr, int low, int high) { }      
\end{lstlisting}

Note that integers alpha and beta are cutoff values for using isort, pick\_pivot3, or pick\_pivot9. Besides adding the test is\_sorted and selecting better pivots, this version of \qusort is different from the textbook version of \qusort~\cite{10.5555/1614191} in that the position of the pivot element is not identified and there is no attempt to move the pivot element in between the two subarrays after the splitting. 
\end{comment}
\end{document}
